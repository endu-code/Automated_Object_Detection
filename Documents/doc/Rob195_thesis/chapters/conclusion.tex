\chapter{Conclusion}
\label{chap:conclusion}

A basic approach for a collision avoidance system for collaborative robots has been programmed and partially successfully implemented. 

The advantages and disadvantages of necessary tools like cameras and libraries have been evaluated and tested on a basic use case. 
For industrial use a better performing and more extended camera set up needs to be implemented, so the workspace can be monitored adequately.

The open source libraries used for point cloud processing and occupancy grid mapping provide the needed algorithms for this project.

The programmed use case is very basic but enables to create the most common movements of an industrial application and is therefore sufficient for this project. More important is the communication from robot to system and back. The Robots position can be sent to the system and the robots movement can be defined externally and extended with gripping commands.

Besides the movement of the robot, the data gathering of the robots workspace is key in this project. Cameras need to be mounted in the right positions to avoid creating shadows over the obstacles in the workspace and thus preventing them to be detected.




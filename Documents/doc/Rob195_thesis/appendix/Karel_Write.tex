\lstset{language=Pascal,
	basicstyle=\ttfamily,
	numbers=left,
	stepnumber=1,
	keywordstyle=\color{blue},
	stringstyle=\color{red},
	commentstyle=\color{green},
	morecomment=[l][\color{magenta}]{\#}
}     

\chapter{KAREL Write - code}
\label{app:karelwrite}
\begin{lstlisting}[frame = single, label={lst:karelread}]  
PROGRAM test_write_c
%STACKSIZE = 4000
%NOLOCKGROUP
%NOPAUSE=ERROR+COMMAND+TPENABLE
%ENVIRONMENT uif
%ENVIRONMENT sysdef
%ENVIRONMENT memo
%ENVIRONMENT kclop
%ENVIRONMENT bynam
%ENVIRONMENT fdev
%ENVIRONMENT flbt
%ENVIRONMENT REGOPE
%INCLUDE klevccdf
%INCLUDE klevkeys
%INCLUDE klevkmsk
-------------------------------------------------------------------------------
VAR
file_var   : FILE
tmp_int   : INTEGER
tmp_int1  : INTEGER
tmp_str   : STRING[128]
STATUS  : INTEGER
entry   : INTEGER
n_j_pos : JOINTPOS6     -- position of robot joints to go to, IMPORTANT: JOINTPOSX must be defined, just using JOINTPOS results in uninitialized variable error. X marks the number of Joints.
new_pos : XYZWPR
stat_   : INTEGER     -- status variable
n_real_array : ARRAY[6] OF REAL  -- REAL Array
indx   : INTEGER    --for counter
group_no: INTEGER
CONST

MOVE_PREG = 90			-- POS Register Number wich will be used to store the JointPos
WRITE_FLAG = 199
READ_FLAG = 200
-------------------------------------------------------------------------------
BEGIN
group_no = 1
--Setze $UTOOL $UFRAME mit aktuell im TP eingestellten UT und UF
$GROUP[group_no].$UFRAME = $MNUFRAME[group_no, $MNUFRAMENUM[group_no]]
$GROUP[group_no].$UTOOL = $MNUTOOL[group_no, $MNUTOOLNUM[group_no]]
new_pos = CURPOS(0,0)
--SET_POS_REG(MOVE_PREG, new_pos, STATUS)
FORCE_SPMENU(TP_PANEL,SPI_TPUSER,1)
WRITE TPDISPLAY (CHR(128),CHR(137))
SET_FILE_ATR(file_var, ATR_IA)
-- set the server port before doing a connect
SET_VAR(entry, '*SYSTEM*','$HOSTS_CFG[4].$SERVER_PORT',59003,STATUS)
WRITE TPDISPLAY('Connecting..',CR)
MSG_CONNECT('S4:',STATUS)
WRITE TPDISPLAY(' CONNECT STATUS = ',STATUS,CR)
IF STATUS = 0 THEN
-- Open S4:
WRITE TPDISPLAY('Opening',CR)
--   FOR tmp_int1 = 1 TO 20 DO
OPEN FILE file_var ('rw','S4:')
STATUS = IO_STATUS(file_var)
WRITE TPDISPLAY(STATUS,CR)
IF STATUS = 0 THEN
-- write an integer
WRITE TPDISPLAY('Reading',CR)

-- Read 10 bytes

FOR indx = 1 TO 6 DO
BYTES_AHEAD(file_var, entry, STATUS)
READ file_var (tmp_str::126)
WRITE TPDISPLAY('receive:',tmp_str,CR )
CNV_STR_REAL(tmp_str, n_real_array[indx])
--              n_real_array[indx] = tmp_str
--		READ file_var (n_real_array[indx]::126,CR)
WRITE TPDISPLAY('2.receive:',n_real_array[indx],CR )
ENDFOR
new_pos.X = n_real_array[1]
new_pos.Y = n_real_array[2]
new_pos.Z = n_real_array[3]
new_pos.W = n_real_array[4]
new_pos.P = n_real_array[5]
new_pos.R = n_real_array[6]
WRITE TPDISPLAY('New Positions:',new_pos, CR)
SET_POS_REG(MOVE_PREG, new_pos, STATUS)

CLOSE FILE file_var
ENDIF

WRITE TPDISPLAY('Disconnecting..',CR)
MSG_DISCO('S4:',STATUS)
WRITE TPDISPLAY('Done.',CR)
SET_INT_REG(WRITE_FLAG, 0, STATUS)
SET_INT_REG(READ_FLAG, 0, STATUS)
ENDIF
END test_write_c
\end{lstlisting}
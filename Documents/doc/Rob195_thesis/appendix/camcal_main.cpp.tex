\chapter{camCalibration main.cp \- code}
\label{app:camCalibration}

\lstset{language=C++,
	numbers=left,
	stepnumber=1,
	basicstyle=\ttfamily,
	keywordstyle=\color{blue},
	stringstyle=\color{red},
	commentstyle=\color{green},
	morecomment=[l][\color{magenta}]{\#}
}

\begin{lstlisting}[frame = single, label={lst:cppread1}]
#include <sys/socket.h>
#include <netinet/in.h>
#include <arpa/inet.h>
#include <cstring>
#include <iostream>
#include <stdlib.h>
#include <string>
#include <math.h>
#include <chrono>

#include <pcl/io/openni2_grabber.h>
#include <pcl/visualization/cloud_viewer.h>
#include <pcl/io/pcd_io.h>
#include <pcl/io/png_io.h>
#include <pcl/io/ply_io.h>
#include <pcl/point_types.h>
#include <pcl/console/parse.h>
#include <pcl/common/transforms.h>
#include <pcl/point_types.h>
#include <pcl/filters/passthrough.h>
#include <pcl/compression/octree_pointcloud_compression.h>
#include <pcl/filters/voxel_grid.h>
bool camAsaved = false;
bool camBsaved = false;


using namespace std;
using namespace pcl;


class SimpleOpenNIViewer
{
public:
SimpleOpenNIViewer () : viewer ("PCL OpenNI Viewer") {}
void cloud_cb_2_ (const pcl::PointCloud<pcl::PointXYZRGBA>::ConstPtr &cloud)
{
if (!viewer.wasStopped()){

pcl::PassThrough<pcl::PointXYZRGBA> pass;

pcl::PointCloud<pcl::PointXYZRGBA>::Ptr CloudCamBfiltered (new pcl::PointCloud<pcl::PointXYZRGBA> ());

pass.setInputCloud (cloud);
pass.setFilterFieldName ("z");
pass.setFilterLimits (0.0, 3.5);
//pass.setFilterLimitsNegative (true);
pass.filter (*CloudCamBfiltered);


viewer.showCloud (CloudCamBfiltered, cloudnameB);
stringstream stream;
stream << "referenceCloudB.pcd";
string filename = stream.str();
io::savePCDFileASCII(filename, *CloudCamBfiltered);

stream.str(std::string());
stream << "referenceCloudB.png";
filename = stream.str();
io::savePNGFile(filename, *CloudCamBfiltered, "rgb");
}
}

void cloud_cb_ (const pcl::PointCloud<pcl::PointXYZRGBA>::ConstPtr &cloud)
{
if (!viewer.wasStopped()){
pcl::PassThrough<pcl::PointXYZRGBA> pass;

pcl::PointCloud<pcl::PointXYZRGBA>::Ptr CloudCamAfiltered (new pcl::PointCloud<pcl::PointXYZRGBA> ());

pass.setInputCloud (cloud);
pass.setFilterFieldName ("z");
pass.setFilterLimits (0.0, 4.1);
//pass.setFilterLimitsNegative (true);
pass.filter (*CloudCamAfiltered);

viewer.showCloud (CloudCamAfiltered, cloudnameA);
stringstream stream;
stream << "referenceCloudA.pcd";
string filename = stream.str();
io::savePCDFileASCII(filename, *CloudCamAfiltered);

stream.str(std::string());
stream << "referenceCloudA.png";
filename = stream.str();
io::savePNGFile(filename, *CloudCamAfiltered, "rgb");
}
}

void run ()
{
//        //**********************************
//        start first Xtion
Grabber* interface = new io::OpenNI2Grabber("#1");

boost::function<void (const pcl::PointCloud<pcl::PointXYZRGBA>::ConstPtr&)> f =
boost::bind (&SimpleOpenNIViewer::cloud_cb_, this, _1);

interface->registerCallback (f);
interface->start();


//**********************************
//**********************************
//        start second Xtion
Grabber* interface2 = new io::OpenNI2Grabber("#2");
boost::function<void (const pcl::PointCloud<pcl::PointXYZRGBA>::ConstPtr&)> f2 =
boost::bind (&SimpleOpenNIViewer::cloud_cb_2_, this, _1);

interface2->registerCallback (f2);
interface2->start();
//**********************************

while (!viewer.wasStopped())
{

}
interface->stop ();
interface2->stop ();
}

pcl::visualization::CloudViewer viewer;
const std::string cloudnameA = "cloudA";
const std::string cloudnameB = "cloudB";
};


int main ()
{

int tableLength = 2250;
int tableWidth  = 1480;
float tableHeight = 670;

float tableStartingpointX = -1274.883;
float tableStartingpointY = 1154.362;
float tableStartingpointZ = -653.149;

// Fill in the cloud data

pcl::PointCloud<pcl::PointXYZRGBA>::Ptr basic_cloud_ptr (new pcl::PointCloud<pcl::PointXYZRGBA>);
uint8_t r(255), g(15), b(15);
uint8_t r3(15), g3(15), b3(255);
for (float xAxis = tableStartingpointX; xAxis < tableStartingpointX+tableLength; xAxis += 10){
for(float yAxis = tableStartingpointY; yAxis < tableStartingpointY+tableWidth; yAxis += 10){
//for(float zAxis = tableStartingpointZ; zAxis < tableStartingpointZ+tableHeight; zAxis += 10){
pcl::PointXYZRGBA basic_point;
basic_point.x = xAxis/1000;
basic_point.y = yAxis/1000;
basic_point.z = tableStartingpointZ/1000;


if(xAxis <= tableStartingpointX+1000){
if(yAxis <= tableStartingpointY+1000){
uint32_t rgb = (static_cast<uint32_t>(r3) << 16 |
static_cast<uint32_t>(g3) << 8 | static_cast<uint32_t>(b3));
basic_point.rgb = *reinterpret_cast<float*>(&rgb);
}
else{
uint32_t rgb = (static_cast<uint32_t>(r) << 16 |
static_cast<uint32_t>(g) << 8 | static_cast<uint32_t>(b));
basic_point.rgb = *reinterpret_cast<float*>(&rgb);
}
}
else if(xAxis > (tableStartingpointX+tableLength) -1000){
if(yAxis > (tableStartingpointY+tableWidth)-1000){
uint32_t rgb = (static_cast<uint32_t>(r3) << 16 |
static_cast<uint32_t>(g3) << 8 | static_cast<uint32_t>(b3));
basic_point.rgb = *reinterpret_cast<float*>(&rgb);
}
else {
uint32_t rgb = (static_cast<uint32_t>(r) << 16 |
static_cast<uint32_t>(g) << 8 | static_cast<uint32_t>(b));
basic_point.rgb = *reinterpret_cast<float*>(&rgb);
}

}
else{
uint32_t rgb = (static_cast<uint32_t>(r) << 16 |
static_cast<uint32_t>(g) << 8 | static_cast<uint32_t>(b));
basic_point.rgb = *reinterpret_cast<float*>(&rgb);
}


basic_cloud_ptr->points.push_back(basic_point);


//}
}
}


uint8_t r2(15), g2(255), b2(15);
for (float xAxis = tableStartingpointX; xAxis < tableStartingpointX+500; xAxis += 10){
for(float yAxis = tableStartingpointY; yAxis < tableStartingpointY+500; yAxis += 10){
//for(float zAxis = tableStartingpointZ; zAxis < tableStartingpointZ+tableHeight; zAxis += 10){
pcl::PointXYZRGBA basic_point;
basic_point.x = xAxis/1000;
basic_point.y = yAxis/1000;
basic_point.z = (tableStartingpointZ - 100)/1000;
uint32_t rgb = (static_cast<uint32_t>(r2) << 16 |
static_cast<uint32_t>(g2) << 8 | static_cast<uint32_t>(b2));
basic_point.rgb = *reinterpret_cast<float*>(&rgb);
basic_cloud_ptr->points.push_back(basic_point);

//}
}
}

basic_cloud_ptr->width = (int) basic_cloud_ptr->points.size ();
basic_cloud_ptr->height = 1;

pcl::io::savePCDFileASCII ("referenceCloudTable.pcd", *basic_cloud_ptr);
std::cerr << "Saved data points to referenceCloudTable.pcd." << std::endl;

SimpleOpenNIViewer v;
v.run ();

return (0);
}

\end{lstlisting}

\chapter{Introduction}
\label{chap:introduction}
\setstretch{1.5}
Collaborative robots are meant to be flexible and easy to be reprogrammed so that they can be used efficiently in the modern industrial environment. One of the most important aspects for collaborative robots is safety. The robot shall not cause any harm to people or material. Currently, most Systems rely on force or torque sensors to stop any motion if a collision is taking place. This means, that the robot motion only stops when a collision already happened. 
This approach works reasonably well for human-machine collisions, but in a collaborative situation, the robot acts within a dynamic workspace. Objects like containers with liquids or heavy and unstable objects may obstruct the workspace. These objects may be tipped over by the robot without triggering a halt based on the force sensors, which could pose a health risk.
In addition, halts caused ba the force sensors cause downtime in the production process which leads to higher production costs and longer production times, which companies like to avoid.
The goal of this thesis is to develop a vision system, which detects in real-time any occupied area in the workspace, that means people or objects which are within reach of the robot, and adapts the robots trajectories to avoid any collisions, thus providing an extra layer of safety. This would allow the robot to work in a modifiable workspace together with a human and to adapt his movements according to the human ones \cite{work_desc}. 


